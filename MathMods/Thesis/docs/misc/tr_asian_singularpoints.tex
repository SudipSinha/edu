%%%%%%%%%%%%%%%%%%%%%%%%%%%%%%%%%%%%%%%%%
% Short Sectioned Assignment
% LaTeX Template
% Version 1.0 (5/5/12)
%
% This template has been downloaded from:
% http://www.LaTeXTemplates.com
%
% Original author:
% Frits Wenneker (http://www.howtotex.com)
%
% License:
% CC BY-NC-SA 3.0 (http://creativecommons.org/licenses/by-nc-sa/3.0/)
%
%%%%%%%%%%%%%%%%%%%%%%%%%%%%%%%%%%%%%%%%%

%----------------------------------------------------------------------------------------
%	PACKAGES AND OTHER DOCUMENT CONFIGURATIONS
%----------------------------------------------------------------------------------------
\documentclass[a4paper]{article}

\usepackage[hmarginratio=1:1,top=32mm,margin=20mm]{geometry} % Document margins
\usepackage{lmodern}
\usepackage[T1]{fontenc}  % Use 8-bit encoding that has 256 glyphs
\usepackage[english]{babel}  % English language/hyphenation
%\usepackage[utf8x]{inputenc}
\usepackage{amsmath,amsfonts,amsthm,bm} % AMS - Math, Fonts, Theorems, BoldMath
\usepackage{graphicx} % For putting pictures
\usepackage{booktabs} % Horizontal rules in tables
\usepackage{hyperref} % For hyperlinks in the PDF

\usepackage{fancyhdr} % Headers and footers
\pagestyle{fancy} % All pages have headers and footers
\fancyhead{} % Blank out the default header
\fancyfoot{} % Blank out the default footer
\fancyhead[C]{Thesis} % Custom header text
\fancyfoot[L]{MathMods $\bullet$ 2015 Apr}
\fancyfoot[R]{Page \huge{\thepage}} % Custom footer text

\numberwithin{equation}{section} % Number equations within sections (i.e. 1.1, 1.2, 2.1, 2.2 instead of 1, 2, 3, 4)
\numberwithin{figure}{section} % Number figures within sections (i.e. 1.1, 1.2, 2.1, 2.2 instead of 1, 2, 3, 4)
\numberwithin{table}{section} % Number tables within sections (i.e. 1.1, 1.2, 2.1, 2.2 instead of 1, 2, 3, 4)

\setlength\parindent{0pt} % Removes all indentation from paragraphs - comment this line for an assignment with lots of text

\newtheorem{thm}{Theorem}
\newtheorem{corr}{Corollary}
\newtheorem{lem}{Lemma}
\newtheorem{prop}{Proposition}
\newtheorem*{rem}{Remark}
\newtheorem{defn}{Definition}

%----------------------------------------------------------------------------------------
%	TITLE SECTION
%----------------------------------------------------------------------------------------

\newcommand{\horrule}[1]{\rule{\linewidth}{#1}} % Create horizontal rule command with 1 argument of height

\title{	
\normalfont \normalsize 
\textsc{Thesis} \\ [25pt]
\horrule{0.5pt} \\[0.4cm] % Thin top horizontal rule
\huge The Singular Points method for Asian American options for local volatility models % The assignment title
\horrule{2pt} \\[0.5cm] % Thick bottom horizontal rule
}

\author{Sudip Sinha}

\date{\normalsize\today}

\begin{document}

\maketitle  % Print the title

%\begin{abstract}
%Your abstract.
%\end{abstract}

%----------------------------------------------------------------------------------------
%	PROBLEM 1
%----------------------------------------------------------------------------------------


\section{Introduction}
The Singular Points method for Asian American options for local volatility models.


\section{Notations}
$[n] = \{0, 1, 2, \dots, n\}$


\section{Basic formulae}

\begin{subequations} \label{eq:arithmeticmean}
Arithmetic average
\begin{align}
A_{n} &= \frac{\sum_{i=0}^n S_i}{n+1} \\
\implies (n+1) A_{n} &= \sum_{i=0}^n S_i
\end{align}
\end{subequations}


\section{Results}

\begin{defn}[Path]
A path is a sequence $(j_i)_{i \in [n]}$ such that $j_{i+1} \in \{ j_i,j_i+1 \}$.
\end{defn}


\begin{lem}
Let there be two paths $\alpha$ and $\beta$, such that $S_{i,j_i^\alpha} >= S_{i,j_i^\beta} \; \forall i$. Then $A^\alpha >= A^\beta$.
\end{lem}

\begin{proof}
Clearly if $S_{i,j_i^\alpha} = S_{i,j_i^\beta} \; \forall i$, then $A^\alpha = A^\beta$.

We only need to show the case of inequality.
Let $S_{i,j_i^\alpha} = S_{i,j_i^\beta} \; \forall i \in [n] \setminus \{l\}$. That is, $S_{l,j_l^\alpha} > S_{l,j_l^\beta}$.

Now, from equation \ref{eq:arithmeticmean}, we have:
\begin{align*}
(n+1) A_{n,j}^\alpha &= \sum_{i=0}^{l-1} S_{i,j_i} + S_{l,j_l^\alpha} + \sum_{i=l+1}^{n} S_{i,j_i} \\
(n+1) A_{n,j}^\beta &= \sum_{i=0}^{l-1} S_{i,j_i} + S_{l,j_l^\beta} + \sum_{i=l+1}^{n} S_{i,j_i} \\
\implies (n+1) \left(A_{n,j}^\alpha - A_{n,j}^\beta\right) &= S_{l,j_l^\alpha} - S_{l,j_l^\beta} \\
	&= S_{l-1,j_{l-1}} u_l - S_{l-1,j_{l-1}} d_l \\
	&= S_{l-1,j_{l-1}} (u_l - d_l) > 0 \\
\implies A_{n,j}^\alpha > A_{n,j}^\beta
\end{align*}

\end{proof}


\begin{rem}
The path $\alpha$ signifies the path above and $\beta$ signifies the path below. Thus, the path above always has a higher arithmetic mean.
\end{rem}


\begin{corr}
At each node $N(n,j)$, the average values vary between a minimum average $A_{n,j}^{\mathrm{min}}$ (corresponding to the path with $(n-j)$ down movements followed by $j$ up movements) and a maximum average $A_{n,j}^{\mathrm{max}}$ (corresponding to the path with $j$ up movements followed by $(n-j)$ down movements).
\end{corr}

\begin{proof}
The path 'min' is the bottom-most one and 'max' is the topmost one.
\end{proof}


\begin{lem}[Lemma 3]
The price function at maturity $v_{n,j}$ is convex and piecewise linear.
\end{lem}
\begin{proof}
By construction. See the paper.
\end{proof}


\begin{lem}[Lemma 4]
The price function $v_{i,j}$ is concave and non-linear.
\end{lem}
\begin{proof}
\ref{eq:arithmeticmean}

\end{proof}

\section{Conclusion}
The singular points method may not be used to price Geometric Asian options.

\end{document}

%%% Local Variables:
%%% mode: latex
%%% TeX-master: t
%%% End:
