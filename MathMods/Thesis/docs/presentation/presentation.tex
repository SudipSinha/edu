\documentclass[utf8]{beamer}

\usepackage{beamerthemeblackboard}
\usepackage{graphicx}

\begin{document}

% set handwritten font, necessary packages are loaded in beamerthemeblackboard.sty
\ECFAugie

\title{Pricing exotic path-dependent options}
\subtitle{The Singular Points method}
\date[2015-10-23]{23\textsuperscript{rd} October, 2015}
\institute[MathMods]{MathMods\\{\small Università degli Studi dell'Aquila}}
%\author[Sudip Sinha]{Sudip Sinha\\{\small Supervisor: Prof. Fabio Antonelli}}
\author{
	\begin{minipage}[t]{0.4\textwidth}
		\begin{center}
			\emph{Candidate:}\\
			{\textbf{\textsc{Sudip Sinha}}}\\
			Matricola: 228435
		\end{center}
	\end{minipage}
	\begin{minipage}[t]{0.5\textwidth}
		\begin{center}
			\emph{Supervisor:} \\
			\textsc{Prof. \textbf{Fabio Antonelli}}
		\end{center}
	\end{minipage}
	}


\begin{frame}[plain]
	\maketitle
\end{frame}

\begin{frame}[t]
	\frametitle{Overview}
	\framesubtitle{}
	\begin{itemize}
		\item Introduction
		\item Preliminaries
		\item Proof
		\item Summary
	\end{itemize}
\end{frame}

\begin{frame}[t]
	\frametitle{Introduction}
	The Routh-Hurwitz test enables you to find out if a polynomial is stable without having to find all its roots.
	\begin{equation}
		p(s) = a_n s^n + a_{n-1} s^{n-1} + \ldots + a_1 s + a_0 \,\, (a_i \in \mathbb{R}, s_i \in \mathbb{C})
	\end{equation}
\end{frame}

\begin{frame}[t]
	\frametitle{Example}
	Let's consider a simple polynomial:
	\begin{equation}
		f(s) = x^4 + 3 x^3 - 5 x^2 + x - 7
	\end{equation}
	To find out if this polynomial is stable we need to find its Routh-table.
\end{frame}

\end{document}
