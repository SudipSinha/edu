\message{ !name(SP.tex)}
\message{ !name(SP.tex) !offset(-2) }
\section{Notations}
$[n] = \{0, 1, 2, \dots, n\}$


\section{Guidelines}
\label{sec:guidelines}
\begin{itemize}
\item Understand path-dependent options
\item Discuss existing methods for pricing path-dependent options
\item The Singular Points method
\item Possibilities of extension
\end{itemize}



\section{Introduction}
\label{sec:intro}
As we have seen in the earlier chapters, European and American (path-independent) options may be priced using the CRR and BS models. But for many path depend options, we cannot find a closed-form pricing formula in the BS model. One way to overcome this difficulty is the use of tree or lattice methods in the CRR model. But such methods are slow and very memory intensive owing to the exponential nature of the number of possible paths. Gaudenzi et al%\cite{Gaudenzi2010} introduced a new method for pricing path-path dependent options in an efficient manner.


\section{Existing methods}
\label{sec:existing-methods}


\section{The Singular Points method}
\label{sec:sing-points-meth}





\begin{align} \label{eq:am}
  A_{n} &= \frac{\sum_{i=0}^n S_i}{n+1} \\
  \implies (n+1) A_{n} &= \sum_{i=0}^n S_i
\end{align}


\begin{defn}[Path]
  A path is a sequence $(j_i)_{i \in [n]}$ such that $j_{i+1} \in \{ j_i,j_i+1 \}$.
\end{defn}


\begin{thm}
  Let there be two paths $\alpha$ and $\beta$, such that $S_{i,j_i^\alpha} >= S_{i,j_i^\beta} \; \forall i$. Then $A^\alpha >= A^\beta$.
\end{thm}

\begin{proof}
  Clearly if $S_{i,j_i^\alpha} = S_{i,j_i^\beta} \; \forall i$, then $A^\alpha = A^\beta$.

  We only need to show the case of inequality.
  Let $S_{i,j_i^\alpha} = S_{i,j_i^\beta} \; \forall i \in [n] \setminus \{l\}$. That is, $S_{l,j_l^\alpha} > S_{l,j_l^\beta}$.

  Now, from equation \ref{eq:am}, we have:
  \begin{align*}
    (n+1) A_{n,j}^\alpha &= \sum_{i=0}^{l-1} S_{i,j_i} + S_{l,j_l^\alpha} + \sum_{i=l+1}^{n} S_{i,j_i} \\
    (n+1) A_{n,j}^\beta &= \sum_{i=0}^{l-1} S_{i,j_i} + S_{l,j_l^\beta} + \sum_{i=l+1}^{n} S_{i,j_i} \\
    \implies (n+1) \left(A_{n,j}^\alpha - A_{n,j}^\beta\right) &= S_{l,j_l^\alpha} - S_{l,j_l^\beta} \\
                         &= S_{l-1,j_{l-1}} u_l - S_{l-1,j_{l-1}} d_l \\
                         &= S_{l-1,j_{l-1}} (u_l - d_l) > 0 \\
    \implies A_{n,j}^\alpha > A_{n,j}^\beta
  \end{align*}

\end{proof}


\paragraph{Remark}
The path $\alpha$ signifies the path above and $\beta$ signifies the path below. Thus, the path above always has a higher arithmetic mean.


\begin{thm}
  At each node $N(n,j)$, the average values vary between a minimum average $A_{n,j}^{\mathrm{min}}$ (path corresponding to the path with $(n-j)$ down movements followed by $j$ up movements) and a maximum average $A_{n,j}^{\mathrm{max}}$ (path corresponding to the path with $j$ up movements followed by $(n-j)$ down movements).
\end{thm}


\begin{defn}[Singular points, singular values]
  Let $ P = \{ (x_i, y_i)  \}_{i \in [n]} $, $ n \in \mathbb{N} $ be a set of points such that $ a = x_1 < \dots < x_n < b $ and
  \begin{equation}
    \frac{y_{i} - y_{i-1}}{x_{i} - x_{i-1}} \leq \frac{y_{i+1} - y_{i}}{x_{i+1} - x_{i}}
  \end{equation}
  Then the elements of P are called singular points and the abscissae are called singular values.
\end{defn}


%%% Local Variables:
%%% mode: latex
%%% TeX-master: t
%%% End:

\message{ !name(SP.tex) !offset(-91) }
