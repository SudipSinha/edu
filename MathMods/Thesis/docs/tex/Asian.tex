% !TeX root = ../thesis.tex
% !TeX spellcheck = en_GB
% !TeX encoding = UTF-8


As we have seen in the earlier chapters, European options may be priced using the Cox-Ross-Rubinstein and the Black-Scholes models. Even though the Black-Scholes model has a very high degree of computability, it does not allow us to find closed-form pricing formulae for many path-dependent options, including American options. The way out is by using numerical methods. Numerical methods using discrete time structures were introduced in Chapter \ref{cha:models}. One simple idea is to apply an explicit pricing scheme using a lattice method, which converges to the Black-Scholes model as the number of time steps increase to infinity. But the exponential number of paths ($2^n$ to be exact, where $n$ is the number of time steps) make the method very slow and memory intensive, making it computably impractical. A logical step would be to modify the basic Cox-Ross-Rubinstein model to allow for approximations. In this direction, Gaudenzi et al\cite{Gaudenzi2010} introduced a new method called the \emph{singular points method} for pricing certain path-dependent options in an efficient manner. The chapter is a study on how this method and its applications.

We will mainly focus using the singular points method for pricing Asian options. In Asian options, the price is expressed as a function of some form of averaging on the underlying's price. Popular Asian options use the arithmetic or geometric means as the average. Again, Asian options may be exercised only at maturity (European) or at any time till the maturity (American). They may give the owner of the option the right to either sell (put) or buy (call). Theoretically, we will only study calls, because the framework for puts one may be derived in the exact same way.



\section{Literature Review}
\label{sec:asian-lit-review}

Before we go into the details of the singular points method, we shall look into the pre-existing methods of pricing Asian options, and discuss their advantages and disadvantages briefly. As we remarked in Section \ref{subsec:continuous-other}, Asian options cannot be valued by closed-form formulae in the Black–Scholes model, and their valuation requires the use of numerical methods. Here we consider tree methods for pricing these types of options.

The main barrier to applying the Cox–Ross–Rubinstein method \cite{Cox1979} introduced in Chapter \ref{cha:models} to Asian options with arithmetic averages is the exponential increase in the number of paths that the underlying may take, and this increases the computational difficulty very quickly as we increase the number of time steps.

\paragraph{Tree methods}
Alternative feasible approaches were proposed by Hull and White (1993) \cite{Hull1993}, and Barraquand and Pudet (1996) \cite{Barraquand1996}. The main idea behind their procedures is to restrict the range of all the possible arithmetic averages to a set of representative values. These values are selected in order to span all the possible values of the averages achievable at each node of the tree. The price is then computed by a backward induction procedure in which the prices associated with averages not included in the set of representative values are obtained by interpolation. Both of these methods reduce the computational complexity to $ O(n^3) $, $ n $ being the number of steps. Nevertheless, the advantage of speed is offset by the fact that it is difficult to control the precision of the approximations and the convergence to the continuous value. This was highlighted by Forsyth \emph{et al} \cite{Forsyth2002} in 2002. Forsyth \emph{et al} also proved that a procedure of order $ O(n^{\frac{7}{2}}) $ is necessary in order to assure the convergence of these algorithms.

Later, Chalasani \emph{et al} (1999) \cite{Chalasani1999} proposed a totally different approach, which allowed them to obtain thin upper and lower bounds on the exact Cox–Ross–Rubinstein binomial price for American Asian options. Their method requires a forward procedure and a backward induction. This algorithm significantly increases the precision of the estimates but requires a very large amount of memory and has computational complexity $ O(n^4) $.

\paragraph{PDE based methods}
All of the above were tree methods. More recently, very efficient partial differential equation (PDE)-based methods have been introduced by Vecer (2001) \cite{Vecer2001} and d'Halluin et al (2005) \cite{dHalluin2005}. In Vecer's method, the price of the Asian option is characterized by a simple one-dimensional partial differential equation which could be applied to both continuous and discrete average Asian option. The compuatational complexity is $ O(n^2) $. This approach cannot be applied to American fixed-strike Asian options, which, on the other hand, can be treated using the semi-lagrangian approach of d'Halluin \emph{et al}.

Table \ref{tab:asian-literature-review} briefly reviews the discussion above.
\begin{table}[h]
	\label{tab:asian-literature-review}
	\centering
	\caption{Existing methods}
	% \rowcolors{1}{Burlywood1}{}
	\begin{tabular}{cccl}
		\toprule
		Method  &  Type  &  Complexity  &  Remarks  \\
		\midrule
		Binomial  &  Tree  &  $ 2^n $  &  simple, accurate, convergence to continuous  \\
		Hull \& White  &  Tree  &  $ O(n^3) $  &  accuracy and convergence problems  \\
		Barraquand \& Pudet  &  Tree  &  $ O(n^3) $  &  accuracy and convergence problems  \\
		Chalasani et al  &  Tree  &  $ O(n^4) $  &  thin bounds, but very large memory  \\
		Vecer  &  PDE  &  $ O(n^2) $  &  not universally applicable  \\
		d'Halluin  &  PDE  &  NA  &  more general than Vecer \\
		\bottomrule
	\end{tabular}
\end{table}


A number of these algorithms has been implemented in Premia 13. Premia is a software designed for option pricing, hedging and financial model calibration. It has been developed by the `MathFi' team in INRIA. It is provided with its C/C++ source code and an extensive scientific documentation. More information about Premia can be found at the dedicated  website\footnote{\url{https://www.rocq.inria.fr/mathfi/Premia/}}.


\section{The exact binomial algorithm}
The evolution of the risky asset price $ (S_t)_t $ is governed by the Black-Scholes stochastic differential equation
\begin{subequations}
	\begin{align*}
		\frac{\dif S_t}{S_t}  &=  (r - q) \dif t + \sigma \dif W_t  \\
		S_0  &=  s_0
	\end{align*}
\end{subequations}

The quantities used above are defined as follows.
\begin{description}
	\item[$ S_t $] the price of the underlying risky asset at time $ t $
	\item[$ W_t $] a standard Brownian motion under the risk-neutral probability measure $ P^* $
	\item[$ r > 0 $] the instantaneous interest rate
	\item[$ q \ge 0 $] the continuous divident yield
	\item[$ \sigma > 0 $] the volatility of the risky asset
\end{description}
(Refer also to Equation \ref{eq:continuous-risky-sde-risk-neutral} and Remark \ref{rem:continuous-dividend} in Section \ref{sec:continuous} of Chapter \ref{cha:models}.)

Solving the equation, at any time $ t $, the price of the underlying risky asset is given by
\begin{equation*}
\label{eq:clq-risky-prc}
S_t = s_0 e^{ ( (r - q) - \frac{\sigma^2}{2} )t + \sigma B_t }.
\end{equation*}


\paragraph{Asian options}
Asian options are dependent on the averaging of the underlying
The price of an American Asian option of initial time 0 and maturity T is
\begin{equation}
	P(0, S_0, A_0) = \sup_{\tau \in \mathcal{T}_{[0,T]}}  \E^* \left(  e^{-r \tau} \  h(S_{\tau}, A_{\tau})  \mid  S_0 = s_0, A_0 = s_0  \right) 
\end{equation}
The quantities used in the formula are explained below.
\begin{description}
	\item[$ \mathcal{T}_{[0,T]} $] the set of all stopping times with values in $ [0, T ] $
	\item[$ h $] the payoff function
	\item[$ A_\tau $] the average of the price of the underlying asset over the period $ [0, \tau] $
\end{description}

Let $ K $ denote the strike price. The price function may be one of the following
\begin{description}
	\item[fixed Asian call] $ h = (A_T - K)_+ $
	\item[fixed Asian put] $ h = (K - A_T)_+ $
	\item[floating Asian call] $ h = (S_T - A_T)_+ $
	\item[floating Asian put] $ h = (A_T - S_T)_+ $
\end{description}

Now we consider the discrete model. If the number of time steps in the binomial tree is $ n $, then the corresponding time step is $ \Delta T = \frac{T}{n} $. The lognormal diffusion process $ (S_{i \Delta T})_{i \in [n]} $ is approximated by the Cox–Ross–Rubinstein binomial process (refer Equation \ref{eq:discrete-risky-prod-iid}).
\begin{equation*}
	S_i = s_0 \prod_{j=1}^{i} T_j  \qquad  \forall i \in [n] .
\end{equation*}

As usual, we represent the risk-neutral probability by $ p = \frac{R - d}{u - d} $, where $ u = d^{-1} = e^{\sigma \Delta T} $. We denote the effective rate of interest in each period as $ R \coloneqq e^{r \Delta T} $.

In the Cox–Ross–Rubinstein model, the price at time 0 of the Asian option of the American and European types with payoff function $ h $ is given by $ v(0, s_0, s_0) $, where the functions $ v(i, x, y) $ can be computed by the following backward dynamic programming equations.
\begin{subequations}
	\label{eq:asian-dp-eu}
	\begin{align}
		v(n, x, y)  &=  h(x,y)  \\
		v(i, x, y)  &=  \frac{1}{R} \left(  p v \left( i + 1, x u, \frac{(i + 1) y + x u}{i + 2} \right)  \right.  \\
		&  \qquad  \left. + (1-p) v \left( i + 1, x d, \frac{(i + 1) y + x d}{i + 2} \right)  \right)  \qquad   \forall i \in [n - 1]  \nonumber
	\end{align}
\end{subequations}

In case of Asian options of the American type, we modify the equations accordingly.
\begin{subequations}
	\label{eq:asian-dp-am}
	\begin{align}
		v(n, x, y)  &=  h(x,y)  \\
		v(i, x, y)  &=  \max \left\lbrace  h(x, y), \frac{1}{R} \left(  p v \left( i + 1, x u, \frac{(i + 1) y + x u}{i + 2} \right)  \right. \right. \\
		&  \qquad  \left. \left. + (1-p) v \left( i + 1, x d, \frac{(i + 1) y + x d}{i + 2} \right)  \right)  \right\rbrace  \qquad  \forall i \in [n - 1]  \nonumber
	\end{align}
\end{subequations}

The payoff is a function of the average, which is clearly path-dependent. Thus, the option is path-dependent, and the corresponding price tree is non-recombinant. This makes the classical binomial method infeasible after a small number of steps. Note that the binomial tree for the underlying is always recombinant for constant volatility.


\section{The singular points method}
\label{sec:asian-method}

The price of an Asian option at each instance is a continuous function of the underlying's average. Since the number of paths to a node in a binomial tree is finite, we have that at each node of the underlying's binomial tree, the option price may be represented as a piecewise-linear, continuous, convex function of the average. We shall develop the theoretical idea in this section. In the subsequent section, we shall see that the nature of the function allows us to make approximations with \emph{a priori} error bounds.


\begin{dfn}[singular points and singular values] \label{def:asian-sp}
	Let $ P = (P_i)_{i \in [n]} = ( (x_i, y_i) )_{i \in [n]} $, $ n \in \mathbb{N} $ be a sequence of points such that
	\begin{subequations} \label{eq:asian-conditions}
		\begin{align}
			a =& x_0 < x_1 < \dots < x_{n-1} < x_n = b \\
			\label{eq:asian-condition-slope}
			m_{i+1} :=& \frac{y_{i+1} - y_{i}}{x_{i+1} - x_{i}} \le \frac{y_{i+2} - y_{i+1}}{x_{i+2} - x_{i+1}} = m_{i+2} \qquad \forall i \in \{ 1, \dots, n-1 \}
		\end{align}
	\end{subequations}
	
	Let $ f:[a,b] \to [0, \infty) $ be the function obtained by linear interpolation of the points in $P$. From the definition of $f$ and	 \ref{eq:asian-condition-slope}, the function is continuous, piecewise-linear and convex.
	
	Then, the elements of $P$ are called \emph{singular points of $f$} and the abscissae $ \{ x_i \}_{i \in [n]} $ are called \emph{singular values of $f$}.
\end{dfn}


\begin{rem}
	\label{rem:asian-char}
	We note that the singular points characterise such a function completely. This can be seen from the following representation of the function.
	\begin{equation}
		\label{eq:asian-function-repr}
		f(x) = y_0 + \sum_{i=1}^n [ m_i ( \min \{x_{i}, x \} - \min \{ x_{i-1}, x \} ) ]
	\end{equation}
	Where $ m_{i+1} = \frac{y_{i+1} - y_{i}}{x_{i+1} - x_{i}} $ represents the slope of the function between $ x_{i} $ and $ x_{i+1} $.
\end{rem}

\begin{rem}
	From the conditions \ref{eq:asian-conditions}, we get
	\begin{equation*}
		y_0 < y_1 < \dots < y_{n-1} < y_n
	\end{equation*}
	So it is equivalent to sort points using either abscissae or ordinates.
\end{rem}



\subsection{Upper estimates}
\label{subsec:asian-upper-estimates}

The following lemmas shall provide us with the necessary framework for upper and lower estimates for approximations on the functions generated by singular points.

\begin{lmm}[Upper estimate]
	\label{lmm:asian-upper-estimate}
	Let $ f:[a,b] \to [0, \infty) $ be a continuous, piecewise-linear, convex function characterised by the singular points $ P = ( (x_i, y_i) )_{i \in [n]} $. Then, if a point $ (x_j, y_j), j \in \{ 1, \dots, n-1\} $ is removed from the sequence, the function $ f_u: [a,b] \to [0, \infty) $ obtained by the new sequence $ (P_i)_{i \in [n] \setminus \{ j \}} $ is also continuous, piecewise-linear and convex, and
	\begin{equation}
		f_u(x) \ge f(x) \qquad \forall x \in [a,b]
	\end{equation}
\end{lmm}

\begin{proof}
	By construction, $ \forall x \notin ( x_{j-1} , x_{j+1} ), \; f_u(x) = f(x) $.
	
	Again, by construction, $ \forall x \in ( x_{j-1} , x_{j+1} ), f_u(x) = (1-t) f(x_{j-1}) + t f(x_{j+1}) $, where $ t = \frac{ x - x_{j-1} }{ x_{j+1} - x_{j-1} } $.
	
	Now, we have:
	\begin{alignat*}{9}
		          && x_{j-1}  & <  \qquad x          && <  x_{j+1} \\
		\implies  &&       0  & <  \quad x - x_{j-1} && <  x_{j+1} - x_{j-1} \\
		\implies  &&       0  & <  \frac{ x - x_{j-1} }{ x_{j+1} - x_{j-1} } && <  1 \\
		\implies  &&       0  & <  \qquad t          && <  1
	\end{alignat*}
	
	$f$ is convex $\implies \forall t \in (0,1), \; f( (1-t) x_{j-1} + t x_{j+1} ) < (1-t) f(x_{j-1}) + t f(x_{j+1}) $.
	
	Thus, $ f_u(x) \ge f(x) \; \forall x \in [a,b]$.
\end{proof}

\begin{figure}[h]
	\centering
	
	\definecolor{ffxfqq}{rgb}{1.,0.4980392156862745,0.}
	\definecolor{ffqqqq}{rgb}{1.,0.,0.}
	\definecolor{cqcqcq}{rgb}{0.7529411764705882,0.7529411764705882,0.7529411764705882}
	\definecolor{yqyqyq}{rgb}{0.5019607843137255,0.5019607843137255,0.5019607843137255}
	\definecolor{zzttqq}{rgb}{0.6,0.2,0.}
	\definecolor{eqeqeq}{rgb}{0.8784313725490196,0.8784313725490196,0.8784313725490196}
	
	\begin{tikzpicture}[line cap=round,line join=round,>=triangle 45,x=0.8cm,y=0.8cm]
		\draw [color=eqeqeq,dotted, xstep=1.6cm,ystep=1.6cm] (0.,0.) grid (17.,13.);
		\draw[->,color=black] (0.,0.) -- (17.,0.);
		\foreach \x in {,2.,4.,6.,8.,10.,12.,14.,16.}
		\draw[shift={(\x,0)},color=black] (0pt,2pt) -- (0pt,-2pt);
		\draw[color=black] (16.66408269434807,0.08316246397131088) node [anchor=south west] { x};
		\draw[->,color=black] (0.,0.) -- (0.,13.);
		\foreach \y in {,2.,4.,6.,8.,10.,12.}
		\draw[shift={(0,\y)},color=black] (2pt,0pt) -- (-2pt,0pt);
		\draw[color=black] (0.10395306283371611,12.541470198511039) node [anchor=west] { y};
		\clip(0.,0.) rectangle (17.,13.);
		\draw [line width=1.2pt,color=yqyqyq] (8.,5.)-- (12.,8.);
		\draw [line width=1.2pt,color=yqyqyq] (16.,12.)-- (12.,8.);
		\draw [line width=1.2pt,color=yqyqyq] (8.,5.)-- (4.,3.);
		\draw [line width=1.2pt,color=yqyqyq] (4.,3.)-- (1.,2.);
		\draw [line width=0.4pt,color=cqcqcq] (12.,8.)-- (12.,0.);
		\draw [line width=0.4pt,color=cqcqcq] (8.,5.)-- (8.,0.);
		\draw [line width=0.4pt,color=cqcqcq] (4.,3.)-- (4.,0.);
		\draw [line width=0.4pt,color=cqcqcq] (1.,2.)-- (1.,0.);
		\draw [dash pattern=on 2pt off 2pt,color=ffqqqq] (4.,3.)-- (12.,8.);
		\draw [color=ffxfqq] (8.,5.)-- (8.,5.5);
		\draw [line width=0.4pt,color=cqcqcq] (16.,12.)-- (16.,0.);
		\begin{scriptsize}
			\draw [fill=zzttqq] (16.,12.) circle (1.5pt);
			\draw[color=zzttqq] (16.227479830446466,11.294033238941376) node {$P_5$};
			\draw [fill=zzttqq] (12.,8.) circle (1.5pt);
			\draw[color=zzttqq] (12.19410099249828,7.239863120339971) node {$P_4$};
			\draw [fill=zzttqq] (8.,5.) circle (1.5pt);
			\draw[color=zzttqq] (8.223093992250325,4.474711193293883) node {$P_3$};
			\draw [fill=zzttqq] (4.,3.) circle (1.5pt);
			\draw[color=zzttqq] (4.2105057668688834,2.4788120579824215) node {$P_2$};
			\draw [fill=zzttqq] (1.,2.) circle (1.5pt);
			\draw[color=zzttqq] (1.21665755725786,1.647187418269313) node {$P_1$};
			\draw [fill=zzttqq] (16.,0.) circle (1.5pt);
			\draw[color=zzttqq] (16.18589860531298,0.5452847706494437) node {$x_5$};
			\draw [fill=zzttqq] (12.,0.) circle (1.5pt);
			\draw[color=zzttqq] (12.152519767364792,0.524494154656616) node {$x_4$};
			\draw [fill=zzttqq] (8.,0.) circle (1.5pt);
			\draw[color=zzttqq] (8.160722154550095,0.5037035386637883) node {$x_3$};
			\draw [fill=zzttqq] (4.,0.) circle (1.5pt);
			\draw[color=zzttqq] (4.2105057668688834,0.5037035386637883) node {$x_2$};
			\draw [fill=zzttqq] (1.,0.) circle (1.5pt);
			\draw[color=zzttqq] (1.1958669446911165,0.5660753866422714) node {$x_1$};
			\draw [fill=ffqqqq] (8.,5.5) circle (1.5pt);
			\draw[color=ffqqqq] (7.99439725401615,6.075588624741618) node {$Q_3$};
		\end{scriptsize}
	\end{tikzpicture}
	
	\caption{Upper estimate \\ Illustration of Lemma \ref{lmm:asian-upper-estimate} with $ j = 3 $}
	\label{fig:upper-estimate}
\end{figure}



\subsection{Lower estimates}
\label{subsec:asian-lower-estimates}

\begin{lmm}[Lower estimate]
	\label{lmm:asian-lower-estimate}
	Let $ f:[a,b] \to [0, \infty) $ be a continuous, piecewise-linear, convex function characterised by the singular points $ P = ( (x_i, y_i) )_{i \in [n]} $. Let $ l_{j} $ be the line segment joining points $ P_{j-1} $ and $ P_{j} $. Similarly, let $ l_{j+2} $ be the line segment joining points $ P_{j+1} $ and $ P_{j+2} $. Denote the intersection of the line segments $ l_{j} $ and $ l_{j+2} $ by $ \bar{P} = ( \bar{x}, \bar{y} ) $.
	
	Then the function $ f_d: [a,b] \to [0, \infty) $ characterised by $ (P_0, \dots, P_{j-1}, \bar{P}, P_{j+2}, \dots, P_n) $ is also continuous, piecewise-linear and convex, and
	\begin{equation}
		f_d(x) \le f(x) \qquad \forall x \in [a,b]
	\end{equation}
\end{lmm}

\begin{proof}
	First we show the convexity of $f_d$. We know that $f$ satisfies the property of increasing slopes, that is $ m_{i} \le m_{i+1} \le m_{i+2} $. Since $f_d$ is obtained from $f$ by removing the line segment $l_{j+1}$, for $f_d$ we have that $ m_{i} \le m_{i+2} $, which implies that the function $f_d$ is still convex.
	
	Secondly, to prove the inequality, we may look at the convex function $f$ as if it has been obtained by removing point $ \bar{P} $ from the convex function $f_d$. Then, if $ \bar{x} \in ( x_{j} , x_{j+1} ) $, we have, using Lemma \ref{lmm:asian-upper-estimate}, that $ f_d(x) \le f(x) \qquad \forall x \in [a,b] $.
\end{proof}

\begin{figure}[h]
	\centering
	
	\definecolor{ffxfqq}{rgb}{1.,0.4980392156862745,0.}
	\definecolor{ffqqff}{rgb}{1.,0.,1.}
	\definecolor{cqcqcq}{rgb}{0.7529411764705882,0.7529411764705882,0.7529411764705882}
	\definecolor{yqyqyq}{rgb}{0.5019607843137255,0.5019607843137255,0.5019607843137255}
	\definecolor{zzttqq}{rgb}{0.6,0.2,0.}
	\definecolor{eqeqeq}{rgb}{0.8784313725490196,0.8784313725490196,0.8784313725490196}
	\begin{tikzpicture}[line cap=round,line join=round,>=triangle 45,x=0.8cm,y=0.8cm]
		\draw [color=eqeqeq,dotted, xstep=1.6cm,ystep=1.6cm] (0.,0.) grid (17.,13.);
		\draw[->,color=black] (0.,0.) -- (17.,0.);
		\foreach \x in {,2.,4.,6.,8.,10.,12.,14.,16.}
		\draw[shift={(\x,0)},color=black] (0pt,2pt) -- (0pt,-2pt);
		\draw[color=black] (16.672398939374787,0.08316246397131087) node [anchor=south west] { x};
		\draw[->,color=black] (0.,0.) -- (0.,13.);
		\foreach \y in {,2.,4.,6.,8.,10.,12.}
		\draw[shift={(0,\y)},color=black] (2pt,0pt) -- (-2pt,0pt);
		\draw[color=black] (0.10395306283371619,12.533153952113903) node [anchor=west] { y};
		\clip(0.,0.) rectangle (17.,13.);
		\draw [line width=1.2pt,color=yqyqyq] (8.,5.)-- (12.,8.);
		\draw [line width=1.2pt,color=yqyqyq] (16.,12.)-- (12.,8.);
		\draw [line width=1.2pt,color=yqyqyq] (8.,5.)-- (4.,3.);
		\draw [line width=1.2pt,color=yqyqyq] (4.,3.)-- (1.,2.);
		\draw [line width=0.4pt,color=cqcqcq] (12.,8.)-- (12.,0.);
		\draw [line width=0.4pt,color=cqcqcq] (8.,5.)-- (8.,0.);
		\draw [line width=0.4pt,color=cqcqcq] (4.,3.)-- (4.,0.);
		\draw [line width=0.4pt,color=cqcqcq] (1.,2.)-- (1.,0.);
		\draw [dash pattern=on 2pt off 2pt,color=ffqqff] (8.,5.)-- (10.,6.);
		\draw [dash pattern=on 2pt off 2pt,color=ffqqff] (12.,8.)-- (10.,6.);
		\draw [line width=0.4pt,color=cqcqcq] (10.,6.)-- (10.,0.);
		\draw [color=ffxfqq] (10.,6.)-- (10.,6.5);
		\draw [line width=0.4pt,color=cqcqcq] (16.,12.)-- (16.,0.);
		\begin{scriptsize}
			\draw [fill=zzttqq] (16.,12.) circle (1.5pt);
			\draw[color=zzttqq] (16.23579607547318,11.28571699254424) node {$P_5$};
			\draw [fill=zzttqq] (12.,8.) circle (1.5pt);
			\draw[color=zzttqq] (12.181626624958245,7.231546873942835) node {$P_4$};
			\draw [fill=zzttqq] (8.,5.) circle (1.5pt);
			\draw[color=zzttqq] (8.231410237277029,4.487185562889577) node {$P_3$};
			\draw [fill=zzttqq] (4.,3.) circle (1.5pt);
			\draw[color=zzttqq] (4.1980313993288405,2.4704958115852884) node {$P_2$};
			\draw [fill=zzttqq] (1.,2.) circle (1.5pt);
			\draw[color=zzttqq] (1.2041831897178144,1.6388711718721796) node {$P_1$};
			\draw [fill=zzttqq] (16.,0.) circle (1.5pt);
			\draw[color=zzttqq] (16.19421485033969,0.5369685242523106) node {$x_5$};
			\draw [fill=zzttqq] (12.,0.) circle (1.5pt);
			\draw[color=zzttqq] (12.140045399824759,0.5161779082594828) node {$x_4$};
			\draw [fill=zzttqq] (8.,0.) circle (1.5pt);
			\draw[color=zzttqq] (8.169038399576799,0.4953872922666551) node {$x_3$};
			\draw [fill=zzttqq] (4.,0.) circle (1.5pt);
			\draw[color=zzttqq] (4.1980313993288405,0.4953872922666551) node {$x_2$};
			\draw [fill=zzttqq] (1.,0.) circle (1.5pt);
			\draw[color=zzttqq] (1.1833925771510712,0.5577591402451383) node {$x_1$};
			\draw [fill=zzttqq] (10.,0.) circle (1.5pt);
			\draw[color=zzttqq] (10.22730904368438,0.5161779082594828) node {$x_{34}$};
			\draw [fill=ffqqff] (10.,6.) circle (1.5pt);
			\draw[color=ffqqff] (10.206518431117638,5.443553898559651) node {$Q_{34}$};
			\draw [fill=zzttqq] (10.,6.5) circle (1.5pt);
			\draw[color=zzttqq] (10.040193530583691,7.252337489935663) node {$P_{34}$};
			
		\end{scriptsize}
	\end{tikzpicture}
	
	\caption{Lower estimate \\ Illustration of Lemma \ref{lmm:asian-lower-estimate} with $ j = 3, x_{34} = \bar{x}, P_{34} = \bar{P} $}
	\label{fig:lower-estimate}
\end{figure}


The lemmas \ref{lmm:asian-upper-estimate} and \ref{lmm:asian-lower-estimate}, will be used later to reduce both the computational complexity and the memory requirement of the algorithm by removing points or edges, effectively simplifying the function.



\subsection{Notations and conventions}
\label{subsec:asian-notations}

In this and subsequent sections, we shall use the convention that $ [n] = \{ 0, 1, 2, \dots, n \} $.

Let the number of time steps be $n$. Let $i$ denote the highlighted time step, and $j$ represent the number of up movements. In this way, we may represent any node by $ N_{i,j} $. For example, in Figure \ref{fig:asian-paths}, the node denoted by $ S_0 u^2 d $ would be represented as $ N_{3,2} $.

The price of the underlying at each node $ N_{i,j} $ is denoted by $ S_{i,j} $. Since there are $j$ up movements, there must be $ i-j $ down movements, and thus
\begin{equation} \label{eq:asian-am-ij}
	S_{i,j} = S_0 u^{j} d^{i-j} = S_0 u^{j} u^{-(i-j)} = S_0 u^{-i+2j} \qquad \forall i \in [n], \ \forall j \in [i]
\end{equation}


\begin{prp}
	The number of paths to a node $ N_{i,j} $ is $ \binom{i}{j} $.
\end{prp}

\begin{proof}
	At each point in a path, we may choose either an up movement or a down movement. To reach node $ N_{i,j} $, we much choose $j$ up movements among $i$ possibilities. The result follows immediately.
\end{proof}


We denote the number of singular points in a node $ N_{i,j} $ by $ L_{i,j} $, where $ L_{i,j} \in \left[ \binom{i}{j} \right] $. The $ l^\mathrm{th} $ average (in ascending order) ($ l \in \{ 1, \dots, L_{i,j} \} $) is denoted by $ A_{i,j}^l $, and the corresponding price by $ P_{i,j}^l $. Thus the singular points characterising the price function are $ ( ( A_{i,j}^l, P_{i,j}^l ) )_{l \in \{ 1, \dots, L_{i,j} \} } $. Note that it is possible that some paths among all the possible paths may give zero as the price for the option. 


\begin{dfn}[singular average and singular price]
	In the particular case of Asian options with arithmetic mean, each $ A_{i,j}^l $ is called a \emph{singular average} and each $ P_{i,j}^l $ is called a  \emph{singular price}.
\end{dfn}


We recall some basic definitions and derive simple results for the maximum and minimum attainable value of the averages on each node.

Let the spot rate of interest be $ r $ (constant) and the compounding be continuous. Then, the effective compounding rate in each time period $ \Delta t $ is given by $ R $ as
\begin{equation}
	\label{eq:R}
	R = e^{r \Delta t}
\end{equation}
We note that the $R$ is not an instantaneous quantity, but one which is constant on an interval of time.


\begin{dfn}[Arithmetic mean]
	The arithmetic mean of a set of numbers $ \{ S_i \}_{i \in [n]} $ is given by:
	\begin{equation}
		\label{eq:am}
		A_{n} = \frac{\sum_{i=0}^n S_i}{n+1}
	\end{equation}
\end{dfn}


\begin{dfn}[Path]
	A path is a sequence $(j_i)_{i \in [n]}$ such that $j_{i+1} \in \{ j_i,j_i+1 \}$.
\end{dfn}

\begin{eg}
	In Figure \ref{fig:asian-paths}, two paths are shown using red/thicker and blue/thick arrows. The other arrows are in grey/thin. The two paths have the same value at maturity, but give different averages.
\end{eg}


\begin{figure}[h]
	% Recombining 4-step binomial tree for Cox-Ross-Rubinstein model
	\begin{tikzpicture}
		\matrix[column sep=10mm,row sep=1mm] (tree){
			& & & & \node[term] (u4) {$S_0u^4$}; \\
			& & & \node[nterm] (u3) {$S_0u^3$}; & \\
			& & \node[nterm] (u2) {$\bm{S_0 u^2}$}; & & \node[term] (u3d) {$S_0u^3d$}; \\
			& \node[nterm] (u) {$ \bm{S_0 u} $}; & & \node[nterm] (u2d) {$\bm{S_0 u^2 d}$};\\
			\node[term] (s) {$ \bm{S_0} $}; & & \node[nterm] (ud) {$ \bm{S_0 u d} $}; & & \node[term] (u2d2) {$ \bm{S_0 u^2 d^2} $ }; \\
			& \node[nterm] (d) {$ \bm{S_0 d} $}; & &	\node[nterm] (ud2) {$\bm{S_0 u d^2}$};\\
			& & \node[nterm] (d2) {$S_0d^2$}; & & \node[term] (ud3) {$S_0ud^3$}; \\
			& & & \node[nterm] (d3) {$S_0d^3$}; & \\
			& & & & \node[term] (d4) {$S_0d^4$}; \\
		};
		% Lines out of s
		\draw[->,red,ultra thick] (s) -- (u) node[midway,above,sloped] {$p_u$};
		\draw[->,blue,thick] (s) -- (d) node[midway,below,sloped] {$p_d$};
		% Lines out of u
		\draw[->,red,ultra thick] (u) -- (u2) node[midway,above,sloped] {$p_u$};
		\draw[->,gray] (u) -- (ud) node[midway,above,sloped] {$p_d$};
		% Lines out of d
		\draw[->,blue,thick] (d) -- (ud) node[midway,below,sloped] {$p_u$};
		\draw[->,gray] (d) -- (d2) node[midway,below,sloped] {$p_d$};
		% Lines out of u2
		\draw[->,gray] (u2) -- (u3) node[midway,above,sloped] {$p_u$};
		\draw[->,red,ultra thick] (u2) -- (u2d) node[midway,above,sloped] {$p_d$};
		% Lines out of ud
		\draw[->,gray] (ud) -- (u2d) node[midway,above,sloped] {$p_u$};
		\draw[->,blue,thick] (ud) -- (ud2) node[midway,below,sloped] {$p_d$};
		% Lines out of d2
		\draw[->,gray] (d2) -- (ud2) node[midway,below,sloped] {$p_u$};
		\draw[->,gray] (d2) -- (d3) node[midway,below,sloped] {$p_d$};
		% Lines out of u3
		\draw[->,gray] (u3) -- (u4) node[midway,above,sloped] {$p_u$};
		\draw[->,gray] (u3) -- (u3d) node[midway,above,sloped] {$p_d$};
		% Lines out of u2d
		\draw[->,gray] (u2d) -- (u3d) node[midway,above,sloped] {$p_u$};
		\draw[->,red,ultra thick] (u2d) -- (u2d2) node[midway,above,sloped] {$p_d$};
		% Lines out of ud2
		\draw[->,blue,thick] (ud2) -- (u2d2) node[midway,below,sloped] {$p_u$};
		\draw[->,gray] (ud2) -- (ud3) node[midway,below,sloped] {$p_d$};
		% Lines out of d3
		\draw[->,gray] (d3) -- (ud3) node[midway,below,sloped] {$p_u$};
		\draw[->,gray] (d3) -- (d4) node[midway,below,sloped] {$p_d$};
	\end{tikzpicture}
	\caption{A 4-step lattice with different paths to the same point}
	\label{fig:asian-paths}
\end{figure}


\begin{thm}[Path inequality]
	\label{thm:asian-up-dn-path}
	Let there be two paths $\alpha$ and $\beta$, such that $S_{i,j_i^\alpha} \ge S_{i,j_i^\beta} \; \forall i$, where $ ( j_i^\alpha )_{i \in [n]} $ and $ ( j_i^\beta )_{i \in [n]} $ denote the paths as defined above. Denote the corresponding averages by $A^\alpha$ and $A^\beta$, respectively. Then $ A^\alpha \ge A^\beta $.
\end{thm}

\begin{proof}
	Clearly if $S_{i,j_i^\alpha} = S_{i,j_i^\beta} \; \forall i$, then $A^\alpha = A^\beta$.
	
	We only need to show the result in the case of strict inequality at one time.
	Let $ S_{i,j_i^\alpha} = S_{i,j_i^\beta} \; \forall i \in [n] \setminus \{l\} $, and $ S_{l,j_l^\alpha} > S_{l,j_l^\beta}$.
	
	Now, from equation \ref{eq:am}, we have:
	\begin{align*}
		(n+1) A_{n,j}^\alpha &= \sum_{i=0}^{l-1} S_{i,j_i} + S_{l,j_l^\alpha} + \sum_{i=l+1}^{n} S_{i,j_i} \\
		(n+1) A_{n,j}^\beta &= \sum_{i=0}^{l-1} S_{i,j_i} + S_{l,j_l^\beta} + \sum_{i=l+1}^{n} S_{i,j_i} \\
		\implies (n+1) \left(A_{n,j}^\alpha - A_{n,j}^\beta\right) &= S_{l,j_l^\alpha} - S_{l,j_l^\beta} \\
												 &= S_{l-1,j_{l-1}} u_l - S_{l-1,j_{l-1}} d_l \\
												 &= S_{l-1,j_{l-1}} (u_l - d_l) > 0 \qquad (u_l > d_l \text{ by definition}) \\
		\implies A_{n,j}^\alpha > A_{n,j}^\beta
	\end{align*}
	
	Iterating this procedure, we obtain the general case.
\end{proof}


\begin{rem}
	The path $\alpha$ signifies a path \emph{above} and $\beta$ a path \emph{below} in the usual depiction of the binomial tree (the up movement shown above the down movement). Thus, a path which never goes below another cannot have a lower arithmetic mean than the other.
\end{rem}


\begin{crr}
	\label{crr:asian-up-dn-path}
	At each node $ N(i,j) $, the following hold:
	\begin{enumerate}
	\item The minimum average possible $ A_{i,j}^{\min} $ is attained by the path corresponding to the path corresponding to the path with $(i-j)$ down movements followed by $j$ up movements, and
		\begin{equation}	\label{eq:asian-Amin}
			A_{i,j}^{\min} = \frac{S_0}{i+1} \left( \frac{1 - d^{i-j+1}}{1-d} + d^{i-j} u \frac{1 - u^{j}}{1-u} \right)
		\end{equation}
	\item The maximum average possible $ A_{i,j}^{\max} $ is attained by the path corresponding to the path with $j$ up movements followed by $(i-j)$ down movements, and
		\begin{equation} \label{eq:asian-Amax}
			A_{i,j}^{\max} = \frac{S_0}{i+1} \left( \frac{1 - u^{j+1}}{1-u} + u^{j} d \frac{1 - d^{i-j-1}}{1-d} \right)
		\end{equation}
	\end{enumerate}
\end{crr}

\begin{proof}
	We show the proof only for the case of the maximum, since the case of the minimum can be shown by using the exact same argument.
	
	From Theorem \ref{thm:asian-up-dn-path}, the result about path with the maximum average holds directly, since there cannot be a path above the one given by $j$ up movements followed by $(i-j)$ down movements.
	
	The subsequent formula may be derived as follows.
	\begin{align*}
		(i+1) A_{i,j}^{\max} &= \underbrace{ ( S_0 + S_0 u + S_0 u^2 + \dots + S_0 u^j ) }_\text{up movement} + \underbrace{ ( S_0 u^j d + S_0 u^j d^2 + \dots + S_0 u^j d^{i-j} ) }_\text{down movement} \\
		&= S_0 ( (1 + u + u^2 + \dots + u^j ) + u^j d ( 1 + d + \dots + d^{i-j-1} ) ) \\
		&= S_0 \left( \sum_{k=0}^j u^k + u^j d \sum_{k=0}^{i-j-1} d^k \right) \\
		&= S_0 \left( \frac{1 - u^{j+1}}{1-u} + u^{j} d \frac{1 - d^{i-j-1}}{1-d} \right) \qquad \text{(Geometric series)} \\
		\implies A_{i,j}^{\max} &= \frac{S_0}{i+1} \left( \frac{1 - u^{j+1}}{1-u} + u^{j} d \frac{1 - d^{i-j-1}}{1-d} \right)
	\end{align*}
\end{proof}

Table \ref{tab:asian-notations} summarises the discussion above.

\begin{table}[h]
	\label{tab:asian-notations}
	\centering
	\caption{Summary of notations}
%	\rowcolors{1}{Burlywood1}{}
	\begin{tabular}{cccl}
		\toprule
		Symbol & Range & Formula & Description \\
		\midrule
		$ i $ & $ [ n ] $ & & highlighted time step \\
		$ j $ & $ [ i ] $ & & number of up movements \\
		$ N_{i,j} $ & & & node fixed by $ (i,j) $ \\
		$ S_{i,j} $ & $ [0, \infty) $ & Eq \ref{eq:asian-am-ij} & value of the underlying at node $ N_{i,j} $ \\
		$ L_{i,j} $ & $ \left[ \binom{i}{j} \right] $ & & number of singular points in node $ N_{i,j} $ \\
		$ l $ & $ \{ 1, \dots, L \} $ & & index for points in ascending order of averages \\
		$ A_{i,j}^{\min} $ & $ [0, \infty) $ & Eq \ref{eq:asian-Amin} & minimum average attainable for node $ N_{i,j} $ \\
		$ A_{i,j}^{\max} $ & $ [0, \infty) $ & Eq \ref{eq:asian-Amax} & maximum average attainable for node $ N_{i,j} $ \\
		$ A_{i,j}^l $ & $ \left[ A_{i,j}^{\min}, A_{i,j}^{\max} \right] $ & Eq \ref{eq:am} & $ l^\mathrm{th} $ singular average of node $ N_{i,j} $ \\
		$ P_{i,j}^l $ & & & price corresponding to the average $ A_{i,j}^l $ \\
		$ (A_{i,j}^l, P_{i,j}^l) $ & & & $ l^\mathrm{th} $ singular point of node $ N_{i,j} $ \\
		\bottomrule
	\end{tabular}
\end{table}



\section{Fixed-strike Asian options of the European type}
\label{sec:fixed-strike-eu}

This is the simplest case. For this type of option, the pay-off at maturity is dependent only on (some type of) average $ A_T $ at maturity $ T $ and a fixed constant $ K $, and is given by the function
\begin{equation}
	\label{eq:asian-price-eu-asian-am}
	P_T = (A_T - K)_+
\end{equation}

In each node of the binomial tree, we have a set of possible averages depending on the paths which may be taken to arrive at the node, and prices corresponding to each of those averages. We shall show that these points satisfy condition \ref{eq:asian-conditions}, so they completely characterise the price function. So we consider not only averages that are effectively achievable but all the possible averages between the minimum and maximum realized at that point. This gives us the continuous representation of prices. The intuitive idea is that as the time step is reduced to zero, this function converges to the price function of the continuous time model.

We start with the prices at maturity, and proceed using backward iteration. The exact details are explained below.


\paragraph{At maturity ($ i = n $)}

From equations \ref{eq:asian-Amin} and \ref{eq:asian-Amax}, putting $i = n$, we get
\begin{align*}
	A_{n,j}^{\min} &= \frac{S_0}{n+1} \left( \frac{1 - d^{n-j+1}}{1-d} + d^{n-j} u \frac{1 - u^{j}}{1-u} \right) \\
	A_{n,j}^{\max} &= \frac{S_0}{n+1} \left( \frac{1 - u^{j+1}}{1-u} + u^{j} d \frac{1 - d^{n-j-1}}{1-d} \right)
\end{align*}

In defining the price function, we note that three cases may arise.
\begin{itemize}
\item $ j \in \{ 0, n \} $ \\
	In this case, there can be only one path to these nodes, so there is only one average, implying one price and one singular point.
	
\item $ j \notin \{ 0, n \} $ and $ K \in ( A_{n,j}^{\min}, A_{n,j}^{\max} ) $ \\	
	In this case, the price function is characterised by three singular points ($ L_{i,j} = 3 $), $ ( A_{n,j}^l , P_{n,j}^l )_{l \in \{ 1, 2, 3 \} } $, since we need to compare the averages with the fixed strike price $ K $. The points are as follows.  \\
	\begin{equation} \label{eq:asian-price-maturity-kin}
		\begin{aligned}
			( A_{n,j}^1 , P_{n,j}^1 ) &= ( A_{n,j}^{\min} , 0 ) \\
			( A_{n,j}^2 , P_{n,j}^2 ) &= ( K , 0 ) \\
			( A_{n,j}^3 , P_{n,j}^3 ) &= ( A_{n,j}^{\max} , A_{n,j}^{\max} - K ) \\
		\end{aligned}
	\end{equation} \label{eq:asian-price-maturity-kout}
	
\item $ j \notin \{ 0, n \} $ and $ K \notin ( A_{n,j}^{\min}, A_{n,j}^{\max} ) $ \\
	In this case, the price function is characterised by only two singular points ($ L_{i,j} = 2 $), $ ( A_{n,j}^l , P_{n,j}^l )_{l \in \{ 1, 2 \} } $, which are as follows. \\
	\begin{equation}
		\begin{aligned}
			( A_{n,j}^1 , P_{n,j}^1 ) &= ( A_{n,j}^{\min} , ( A_{n,j}^{\min} - K )_+ ) \\
			( A_{n,j}^2 , P_{n,j}^2 ) &= ( A_{n,j}^{\max} , ( A_{n,j}^{\max} - K )_+ ) \\
		\end{aligned}
	\end{equation}
\end{itemize}

\begin{lmm}[Price function at maturity ($ i < n $)]
	\label{lmm:asian-pr-maturity}
	At each node at maturity, the price function $ { v_{n,j}: \left[ A_{n,j}^{\min}, A_{n,j}^{\max} \right] \to \left[ ( A_{n,j}^{\min} - K )_+ , ( A_{n,j}^{\max} - K )_+ \right] } $ defined as $ v_{n,j}(A) = (A - K)_+ $ is continuous, piecewise-linear and convex.
\end{lmm}
\begin{proof}
	The singular points satisfy the conditions \ref{eq:asian-conditions}. So for each $ A \in \left[ A_{n,j}^{\min}, A_{n,j}^{\max} \right] $, the price function ${ v_{n,j}(A) }$ characterised by the singular points is continuous, piecewise-linear and convex by remark \ref{rem:asian-char}.
\end{proof}



\paragraph{Before maturity ($ i < n $)}

\begin{lmm}[Price function at any node]
	\label{lmm:asian-dsc-expt}
	At any node $ N_{i,j} $, the price function $ v_{i,j}: \left[ A_{i,j}^{\min}, A_{i,j}^{\max} \right] \to [0, \infty) $ is continuous, piecewise-linear and convex.
\end{lmm}

\begin{proof}
	We shall prove this using backward induction, the base case at maturity being true by virtue of Lemma \ref{lmm:asian-pr-maturity}.
	We now consider step $ i = n-1 $. Let $A_u$ and $A_d$ respectively represent the averages after an up and down movement corresponding to an average $A$. From equation \ref{eq:am}, we get
	\begin{subequations}
		\label{eq:asian-av-up-dn}
		\begin{align}
			A_u &= \frac{ (i+1) A + S_0 u^{-i+2j+1} }{ i+1 } \\
			A_d &= \frac{ (i+1) A + S_0 u^{-i+2j-1} }{ i+1 }
		\end{align}
	\end{subequations}
	Since the options is of the European type, applying the no-arbitrage condition, the price function $ v_{i,j}: \left[ A_{i,j}^{\min}, A_{i,j}^{\max} \right] \to [0, \infty) $ is obtained by considering the discounted expectation value.
	\begin{equation}
		\label{eq:asian-dsc-expt}
		v_{i,j}(A) = \frac{1}{R} \left[ \pi v_{i+1,j+1}(A_u) + (1 - \pi) v_{i+1,j}(A_d) \right]
	\end{equation}
	From equation \ref{eq:asian-av-up-dn}, we get that $A_u$ and $A_d$ are linear functions of $A$. Thus, $ v_{i+1,j+1}(A_u) = v_{n,j+1}(A_u)$ and $ v_{i+1,j}(A_d) = v_{n,j}(A_d) $ are piecewise-linear convex continuous functions of $A_u$ and $A_d$ respectively. Thus, $ v_{i+1,j+1} $ and $ v_{i+1,j} $ may be seen as a linear combination of the above functions, and is thus piecewise-linear, convex and continuous itself. Again, from equation \ref{eq:asian-dsc-expt}, we get that $v_{i,j}$ is a convex combination of such functions, and the proof is complete.
	
	We showed that if at time $ i+1 $, if the price function is continuous, piecewise-linear and convex, so is it for time $ i $. Since this is true for $ i+1 = n $, the same logic applied iteratively proves that the functions retain the characteristics for all $ i \in [n] $.
\end{proof}
TODO: Insert picture for this.


\begin{rem}
	From Lemma \ref{lmm:asian-dsc-expt}, we see that the price function may be characterised by singular points.
\end{rem}



\subsection{Evaluation of singular points}
\label{subsec:asian-eu-eval}

The evaluation of singular points for any node $ N_{i,j} $ is done by the following algorithm, which works in a backward fashion in time, starting from the maturity.

We note that for the only influencing nodes for the node $ N_{i,j} $ are $ N_{i+1,j+1} $ and $ N_{i+1,j} $. Thus we need to calculate the price of the option for each singular average belonging to either of these nodes.


\paragraph{Up movement}

First we take each singular average $ A_{i+1,j}^l $ belonging to $ N_{i+1,j} $ and project it to $ N_{i,j} $ via the relation
\begin{equation}
	\label{eq:asian-proj-up}
	B^l = \frac{ ( i+2) A_{i+1,j}^l - S_0 u^{-i+2j-1} }{ i+1 }
\end{equation}
Thus, $ B^l $ is that average which after a down movement of the asset gives us the average $ A_{i+1,j}^l $.

Next, we note that $ B^l $ is an increasing function of $ l $, since a higher average at time step $ i $ would yield a higher average at time $ i+1 $. This in turn implies the following:
\begin{itemize}
\item $ B^{L_{i+1,j}} = A_{i+i,j}^{\max} \; \forall j $
\item $ B^1 \notin \left[ A_{i+i,j}^{\min}, A_{i+i,j}^{\max} \right] \ \forall j \in \{1, \dots, i-1 \} $
\end{itemize}
Each $ B^l \in \left[ A_{i,j}^{\min}, A_{i,j}^{\max} \right] $ becomes the singular average of $ N_{i,j} $.

In this way, we have determined the first coordinate of the singular points. We need to determine the second coordinate, or the prices $ v_{i,j}(B^l) $, $ \forall \left[ A_{i,j}^{\min}, A_{i,j}^{\max} \right] $, in order to determine the singular points completely. The idea is to calculate the discounted expected value of the price corresponding to each average $ B^l $ at $ N_{i,j} $. In order to be able to do this, we need the prices corresponding to the average projected to the node $ N_{i+1,j+1} $.

We consider an up movement of the underlying asset from node $ N_{i,j} $. In this case, $ B^l $ transforms into the average $ B^l_u = \left( (i+1) B^l + S_0 u^{-i+2j+1} \right) / ( i+2 ) $. Clearly, this average cannot belong to the set of averages associated with the node $ N_{i+1,j+1} $. Thus, we need to find the index $s$ such that $ B^l_u \in \left[ A_{i+1,j+1}^{s} , A_{i+1,j+1}^{s+1} \right] $. In the intervals the price function is linear, and thus we have
\begin{equation}
	\label{eq:asian-up-lint}
	v_{i+1,j+1} \left( B^l_u \right) = \frac{ P_{i+1,j+1}^{s+1} - P_{i+1,j+1}^{s} }{ A_{i+1,j+1}^{s+1} - A_{i+1,j+1}^{s} } \left( B^l_u - A_{i+1,j+1}^{s} \right) + P_{i+1,j+1}^{s}
\end{equation}

We follow this up by calculating the price associated with the singular value $ B^l $ by evaluating the discounted expectation value.
\begin{equation}
	\label{eq:asian-up-pr}
	v_{i,j}( B^l ) = \frac{1}{R} \left[ \pi v_{i+1,j+1} \left( B^l_u \right) + (1 - \pi) v_{i+1,j} \left( A_{i+1,j}^l \right) \right]
\end{equation}

Figure \ref{fig:asian-2tr-up} depicts the idea.
\begin{figure}[h]
	\begin{tikzpicture}
	\matrix (tree) [column sep=25mm, row sep=1mm]{
		\node[header] (t0) {$ t = i $};  &  \node[header] (t1) {$ t = i+1 $}; \\
		&  \node[term] (u) {$ B^l_u $}; \\
		\node[term] (s) {$ B^l $};  &  \\
		&  \node[term] (d) {$ A_{i+1,j}^l $}; \\
	};
	\draw[->] (s) -- (u) node[midway,above,sloped] {Step 2};
	\draw[->] (d) -- (s) node[midway,below,sloped] {Step 1};
	\end{tikzpicture}
	
	\caption{Up movement}
	\label{fig:asian-2tr-up}
\end{figure}



\paragraph{Down movement}

We now proceed to formulate the theory for the downward movement in the exact same fashion. Define the new average $ C^l $ at the node $ N_{i,j} $ via the relation
\begin{equation}
	\label{eq:asian-proj-dn}
	C^l = \frac{ ( i+2) A_{i+1,j+1}^l - S_0 u^{-i+2j+1} }{ i+1 }
\end{equation}

Again, we note that
\begin{itemize}
\item $ C^1 = A_{i,j}^{\min} \ \forall j $
\item $ C^{L_{i+1,j+1}} \notin \left[ A_{i,j}^{\min}, A_{i,j}^{\max} \right] \ \forall j \in \{1, \dots, i-1 \} $
\item $ C^l_d = \left( (i+1) C^l + S_0 u^{-i+2j-1} \right) / ( i+2 ) $
\end{itemize}
Each $ C^l \in \left[ A_{i,j}^{\min}, A_{i,j}^{\max} \right] $ becomes the singular average of $ N_{i,j} $.

For $ v_{i,j}( C^l ) $, $ \forall \left[ A_{i,j}^{\min}, A_{i,j}^{\max} \right] $, we now have the following.
\begin{equation}
	\label{eq:asian-dn-lint}
	v_{i+1,j+1} \left( C^l_d \right) = \frac{ P_{i+1,j}^{s+1} - P_{i+1,j}^{s} }{ A_{i+1,j}^{s+1} - A_{i+1,j}^{s} } \left( C^l_d - A_{i+1,j}^{s} \right) + P_{i+1,j}^{s}
\end{equation}

\begin{equation}
	\label{eq:asian-dn-pr}
	v_{i,j}( C^l ) = \frac{1}{R} \left[ \pi v_{i+1,j+1} \left( A_{i+1,j+1}^l \right) + (1 - \pi) v_{i+1,j} \left( C^l_d \right) \right]
\end{equation}

Figure \ref{fig:asian-2tr-dn} depicts the idea.
\begin{figure}[h]
	\begin{tikzpicture}
	\matrix (tree) [column sep=25mm, row sep=1mm]{
		\node[header] (t0) {$ t = i $};  &  \node[header] (t1) {$ t = i+1 $}; \\
		&  \node[term] (u) {$ A_{i+1,j+1}^l $}; \\
		\node[term] (s) {$ C^l $};  &  \\
		&  \node[term] (d) {$ C^l_d $}; \\
	};
	\draw[->] (u) -- (s) node[midway,above,sloped] {Step 1};
	\draw[->] (s) -- (d) node[midway,below,sloped] {Step 2};
	\end{tikzpicture}
	
	\caption{Down movement}
	\label{fig:asian-2tr-dn}
\end{figure}


\paragraph{Aggregation}

Now we have the singular points for both up and down movements. We sort these points in ascending order of the first coordinate, i.e. the averages $ B^l $ and $ C^l $ that belong to $ \left[ A_{i,j}^{\min}, A_{i,j}^{\max} \right] $. These is an exhaustive list of all the singular points in the node (by construction). We note that $ L_{i,j} \le L_{i+1,j} + L_{i+1,j+1} - 2 $.

This procedure is applied to all nodes, starting from maturity and proceeding backwards. At the `edge' nodes $ N_{i,0} $ and $ N_{i,i} $, there is only one singular point whose price is given as follows
\begin{subequations}
	\label{eq:asian-terminal-nodes}
	\begin{align}
		P_{i,0}^1 &= \frac{1}{R} \left[ \pi P_{i+1,0}^1 + (1 - \pi) P_{i+1,1}^1 \right] \\
		P_{i,i}^1 &= \frac{1}{R} \left[ \pi P_{i+1,i+1}^1 + (1 - \pi) P_{i+1,i}^{L_{i+1,i}} \right]
	\end{align}
\end{subequations}

Thus we have a complete description of the price function at each node of the binomial tree. The price $ P_{0,0}^1 $ is exactly the binomial price relative to the tree with $n$ steps of a fixed-strike European call option.


\section{Fixed-strike Asian options of the American type}
\label{sec:fixed-strike-am}



TODO
\clearpage
\section{The program}
\label{sec:asian-program}

\subsection{Algorithm}

\begin{algorithm}[H]
	\DontPrintSemicolon
	
	\KwIn{\\
		\qquad \emph{Contract details}  \\
		\qquad \quad time to maturity: $ T $ \\
		\qquad \quad number of observations: $ N $ \\
		\qquad \quad local floor and cap: $ F_{loc}, C_{loc} $  \\
		\qquad \quad global floor and cap: $ F_{glob}, C_{glob} $ \\
		
		\qquad \emph{Details of the underlying asset}  \\
		\qquad \quad initial price: $ s_0 $  \\
		\qquad \quad volatility: $ \sigma $  \\
		\qquad \quad continuous dividend rate: $ q $  \\
		
		\qquad \emph{Market parameters} -- spot interest rate: $ r $ \\
		
		\qquad \emph{Computational parameters} -- time steps within each observed period: $ m $ \\
	}
	
	\KwOut{The price of the option at the initial time}
	
	\Begin{
		Update $ F_{glob} $ and $ C_{glob} $ using Equations \ref{eq:asian-update-glob}. \;
		
		Set $ \Delta T, u, p $ from the formulae in Section \ref{sec:asian-models}. \;
		
		Compute the returns and probabilities using Equation \ref{eq:asian-return-final} and Equation \ref{eq:asian-prb-binom}. \;
		
		Compute $ j_{\min}, j_{\max}, j_0 $ (Equations \ref{eq:asian-j-mix-max}) and shifted returns (Table \ref{tab:asian-shift}). \;
		
		\tcp{$ S_i $ and $ S_+ $ denotes the current ($ i^{\mathrm{th}} $) and next ($ (i+1)^{\mathrm{th}} $) list of singular points.}
		
		$ S_N \leftarrow \{ (N F_{loc}, F_{glob}), (F_{glob}, F_{glob}), (C_{glob}, C_{glob}), (N C_{loc}, C_{glob}) \} $ \tcp{Table \ref{tab:asian-maturity}.}
		
		$ S_+ \leftarrow S_N $ \;
		
		\For{$ i \in \{ N-1, \dots, 0 \} $}{
			$ S_i \leftarrow \emptyset, \qquad L_+ \leftarrow \mathrm{length}(S_+) $ \;
			
			\ForAll{$ (l, j) \in [L_+] \times [j_0 + 1] $}{
				Compute $ B_{l,j} $ using Equation \ref{eq:asian-b}, replacing $ N $ by $ i $. \;
				
				\If{ $ ( B_{l,j} + R_k' ) \notin [ Z_N^{l_0}, Z_N^{l_0 + 1} ] $ } {
					Continue \tcp*{to the next item in the loop}
				}
				
				Find $ l_0 $ such that $ ( B_{l,j} + R_k' ) \in [ Z_{i+1}^{l_0}, Z_{i+1}^{l_0 + 1} ] $. \;
				
				Evaluate $ V_{i+1} (B_{l,j} + R_k') $ by linear interpolation of the extrema. \;
				
				Evaluate $ V_{i}(B_{l,j}) $ by using Equation \ref{eq:asian-penultimate}, replacing $ N $ by $ i $. \;
				
				$ S_i \leftarrow S_i \cup (B_{l,j}, V_{N-1}(B_{l,j})) $ \;
			}
			
			\If{$ m \mid i $}{
				Approximate as described in Section \ref{subsec:asian-approx}. \;
			}
			
			$ S_+ \leftarrow S_i $ \;
		}
		\KwRet{$ (S_i)_{1,2} $ \tcp*{Singular price at time 0} }
	}
	
	\caption{Pricing cliquet options using the singular points method}
\end{algorithm}

\paragraph{Analysis of algorithm}
[TODO if time permits] Consider computational complexity, space complexity.


\subsection{Implementation}
The algorithm was implemented in Python 3.5.0 (2015-09-13).
The system specifications are as follows
[TODO: Shift this to some common region for both Asian and cliquet]

<TODO: The actual implementation goes here>
%\inputminted[tabsize=2]{python}{../code/tr_asian_singularpoints.py}
%\label{lst:asian}



\section{Extensibility}
\label{sec:asian-extensions}

Let us recapitulate the conditions required for the singular points method to work in the case of Asian options with arithmetic mean.
\begin{itemize}
\item The ability to calculate the upper and lower bounds of the mean for all nodes of the tree.
\item The recombinant nature of the tree for the underlying. Note that the tree for the option prices are \emph{not} recombinant.
\item Convexity and piecewise-linearity of the price function on the mean of the underlying.
\item Fixed volatility
\end{itemize}

Keeping these in mind, let us look at the possibility of extending the singular points method to the following cases:
\begin{enumerate}
\item Asian options with geometric mean and fixed volatility.
\item Asian options with arithmetic mean and local volatility.
\end{enumerate}



\subsection{Geometric mean and fixed volatility}
\label{subsec:gm-fixed-vol}

In the case of geometric options, we have a closed form formula under the Black-Scholes market model. We try to extend the singular points method.

Firstly, we show that the result about the maximum and minimum paths still hold in the geometric case.

\begin{dfn}[Geometric mean]
	The geometric mean of the risky asset's prices $ (S_i)_{i \in [n]} $ is given by:
	\begin{equation}
		\label{eq:gm}
		G_{n} = \left( \prod_{i=0}^n S_i \right) ^{\frac{1}{n+1}}
	\end{equation}
\end{dfn}


\begin{lmm}
	At each node $N(i,j)$, the following hold:
	\begin{enumerate}
	\item The maximum average possible $ G_{i,j}^{\max} $ is attained by the path corresponding to the path with $j$ up movements followed by $(i-j)$ down movements.
	\item The minimum average possible $ G_{i,j}^{\min} $ is attained by the path corresponding to the path corresponding to the path with $(i-j)$ down movements followed by $j$ up movements.
	\end{enumerate}
\end{lmm}

\begin{proof}
	The proof is the same as \ref{crr:asian-up-dn-path}, with $A$ replaced by $G$ and relevant modifications.
\end{proof}


One of the central ideas behind the singular points method is that the price of the option is a convex, piecewise-linear function of the average $A$. But in the geometric case, this no longer holds true. For example, take a node $N_{i,j}$ with $ i = n-1 $. The price function given by $ v_{i,j}(G) $, with $ G \in [G^{min},G^{max}] $, can be calculated by the discounted expectation value.
\begin{align}
	v_{i,j}(G) &= \frac{1}{R} \left[ p v_{i+1,j+1}(G_u) + (1-p) v_{i+1,j}(G_d) \right] \\
	G_u &= \left( G^{i+1} S_0 u^{-i+2j+1} \right)^{\frac{1}{i+2}} \propto G^{\frac{i+1}{i+2}} \\
	G_d &= \left( G^{i+1} S_0 u^{-i+2j-1} \right)^{\frac{1}{i+2}} \propto G^{\frac{i+1}{i+2}}
\end{align}
Clearly, the final function $ v_{i,j} $ is not linear in $G$. Rather it is piecewise-concave. Thus we cannot use the singular points method in this case.
TODO: Insert a graph of the function here.



\subsection{Arithmetic mean with local volatility}
\label{subsec:am-local-vol}

In this case, the tree for the underlying is not recombinant, so we do not have more than one singular point in one (non-recombining) node. Clearly, we cannot use the singular points method.



\section{Conclusion}
\label{sec:asian-adv}

We conclude the chapter by noting the pros and cons of the singular points method.
\paragraph{Advantages}
\begin{itemize}
\item Fast -- Experimental order of complexity = $ O(n^3) $
\item It allows us to specify an \emph{a priori} error bound.
\end{itemize}


\paragraph{Disadvantages}
\begin{itemize}
\item Very specific method -- only applicable to a few specific cases.
\end{itemize}


%%% Local Variables:
%%% mode: latex
%%% TeX-master: t
%%% End:
