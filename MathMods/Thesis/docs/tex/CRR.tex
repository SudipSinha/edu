% !TeX root = ../thesis.tex
% !TeX spellcheck = en_GB
% !TeX encoding = UTF-8


\section{Introduction}
\label{sec:crr-intro}
.... Since continuous models are mathematically more complex than their discrete time counterparts (also known as `lattice models'), we shall discuss the latter first. We shall then show that under certain convergence conditions, the discrete models discusses converges to the continuous models.

\section{The binomial model}
\label{sec:crr-binom-model}

We start our discussion with one of the simplest model used for pricing of assets, the binomial model. This model was first introduced by Cox, Ross and Rubinstein \cite{Cox1979} in the paper titled ``Option pricing: A simplified approach'' in 1979. Even though it is very simple, it does contain all the necessary ingredients to construct a viable market model, and to solve the problems of pricing and hedging of derivatives.

In this model, we essentially have
\begin{itemize}
	\item two time points, $ t = 0 $ (present) and $ t = 1 $ (future)
	\item two traded assets
	\begin{itemize}
		\item the riskless asset, usually a bond, which is compounded at a constant rate of interest $ r > 0 $
		\item the risky asset, usually a stock, which may either go up with a factor $ u $, or down with a factor $ d $.
	\end{itemize}
\end{itemize}
The binomial model is so called because there are two times, two assets and two possible movements of the risky asset.

We denote as $ S_0 $ the value of the risky asset at time 0, and by $ S_1 $ its value at time 1. Firstly, in order to have no arbitrage opportunities, we must have $ d < R < u $, where $ R \coloneqq 1 + r $. Secondly, for fairness, there must exist a probability distribution $ p, (1-p) $ -- signifying the probabilities of the up and down movements -- such that the expected value of the asset remains the same as that of the riskless asset given the same time. (See Figure \ref{fig:crr-2tr-underlying}.)


\begin{figure}
	\begin{tikzpicture}
	\matrix (tree) [column sep=25mm, row sep=1mm]{
		\node[header] (t0) {$ t = 0 $};  &  \node[header] (t1) {$ t = 1 $}; \\
		&  \node[term] (u) {$S_1^u = S_0u$}; \\
		\node[term] (s) {$S_0$};  &  \\
		&  \node[term] (d) {$S_1^d = S_0d$}; \\
	};
	\draw[->] (s) -- (u) node[midway,above] {$p_u = p$};
	\draw[->] (s) -- (d) node[midway,below] {$p_d = 1-p$};
	\end{tikzpicture}
	
	\caption{Binomial tree for the underlying}
	\label{fig:crr-2tr-underlying}
\end{figure}


Thus, we have
\begin{equation*}
	S_1 =
	\begin{cases}
		u S_0 & \quad \text{with probability } p \\
		d S_0 & \quad \text{with probability } (1 - p) \\		
	\end{cases}
\end{equation*}

We may also write $ S_1 = T S_0 $, where $ T $ is a random variable taking values in $ \{ u, d \} $ with associated probability distribution $ (p, 1-p) $.
\begin{equation*}
	T =
	\begin{cases}
		u  & \quad \text{with probability } p \\
		d  & \quad \text{with probability } (1 - p) \\		
	\end{cases}
\end{equation*}


We want $ p \in [0,1] $ such that $ E(S_1) = S_0 R $. Thus
\begin{alignat}{9}
	          &&  E(\frac{S_1}{R}) &= S_0 \\
	\implies  &&  \frac{1}{R} ( p u S_0 + ( 1 - p ) d S_0 )  & =  S_0 \nonumber \\
	\implies  &&  p  & =  \frac{R - d}{u - d}
\end{alignat}

The probability thus obtained is called the \emph{risk neutral probability}, because under this probability, it is equivalent for the investor whether he invests in a risky or a riskless asset. Note that this probability is completely objective as it is determined completely by the parameters $ u $, $ d $ and $ r $.

Now we impose the condition $ p \in [0,1] $ to obtain
\begin{equation}
	\label{eq:crr-feasibility-condition}
	d < R < u
\end{equation}


\paragraph{Pricing a call}

Let us use the above model to price a call. Recall that the pay-off of a call is given by $ h(x) = (x - K)_+ $, where $ K $ is the strike price, a fixed value specified in the contract. Thus, we know the values of the call at maturity. To ensure fairness, we may again write the following Note that for financial viability, we must have $ K \in (S_1^d, S_1^u) $, implying $ c_1^u = (S_1^u - K)_+ = (S_1^u - K) $ and $ c_1^d = (S_1^d - K)_+ = 0 $ (See Figure \ref{fig:crr-2tr-call}).


\begin{figure}
	\begin{tikzpicture}
	\matrix (tree) [column sep=25mm, row sep=1mm]{
		\node[header] (t0) {$ t = 0 $};  &  \node[header] (t1) {$ t = 1 $}; \\
		&  \node[term] (u) {$ c_1^u = (S_1^u - K) $}; \\
		\node[term] (s) {$c_0$};  &  \\
		&  \node[term] (d) {$ c_1^d = 0 $}; \\
	};
	\draw[->] (u) -- (s) node[midway,above] {$p_u = p$};
	\draw[->] (d) -- (s) node[midway,below] {$p_d = 1-p$};
	\end{tikzpicture}
	
	\caption{Binomial tree for the underlying}
	\label{fig:crr-2tr-call}
\end{figure}


\begin{align*}
	c_0 &= E(\frac{c_1}{R}) \\
	    &= p c_1^u + (1-p) c_1^d \\
	    &= p c_1^u \\
	    &= \frac{R - d}{u - d} (u S_0 - K)
\end{align*}
Thus we have been able to price the call uniquely at all times. This is an implication of the completeness of the market, whose randomness is totally characterized by the unique probability measure $ p $.

For the sake of completeness, we comment here that the call is also completely hedgable in the market model.







\section{The Cox-Ross-Rubinstein model}
\label{sec:cox-ross-rubinstein}

In this section we extend the binomial model introduced in Section \ref{sec:crr-binom-model} to a sequence of integer times $ [N] \coloneqq \{ 0, 1, \dots, N \}, \  N \in \mathbb{N} $.

Let $ (\Omega, \mathcal{F}, (\mathcal{F}_n)_n, P) $ be a finite probability space ($ |\Omega| < \infty $), such that $ P(\omega) > 0 \  \forall \omega \in \Omega $, endowed with a filtration $ (\mathcal{F}_n)_n $, such that $ \mathcal{F}_0 $ is trivial ($ \mathcal{F}_0 = \{ \emptyset, \Omega \} $).

We now need to invoke two cornerstone theorems, which will allow us to price the option at any time step $ n $.

\begin{thm}[First Fundamental Theorem of Asset Pricing]
	\label{thm:crr-ftoap1}
	The market model is viable if and only if there exists a probability measure $ P^* $ equivalent to the historic probability measure $ P $ under which the discounted prices of the basic risky assets are martingales.
	
	Mathematically,
	$ \text{Viable market} \iff \exists \  P^* \sim P \text{ such that } E^*( \tilde{S}_{n+1} | \mathcal{F}_n ) = \tilde{S}_{n} $.
\end{thm}

\begin{proof}
	See \cite[page 6, Theorem 1.2.7]{Lamberton1996}
\end{proof}


\begin{thm}[Second Fundamental Theorem of Asset Pricing]
	\label{thm:crr-ftoap2}
	The market model is complete if and only if there exists a \textbf{unique} probability measure $ P^* $ equivalent to the historic probability measure $ P $ under which the discounted prices of the basic risky assets are martingales.
	
	Mathematically,
	$ \text{Complete market} \iff \exists! \  P^* \sim P \text{ such that } E^*( \tilde{S}_{n+1} | \mathcal{F}_n ) = \tilde{S}_{n} $.
\end{thm}

\begin{proof}
	See \cite[page 9, Theorem 1.3.4]{Lamberton1996}
\end{proof}

Augmented by these theorems, we may now seek to price options in this model. If the pay-off of an option at maturity is given by the function $ h(x) $ (for example, in the case of a call, $ h(x) = (x - K)_+ $), then the price of the option at any time step $ n $ is given by $ E^*( h(S_N) | \mathcal{F}_n ) $


\begin{figure}
	\begin{tikzpicture}
	\matrix[column sep=10mm,row sep=1mm] (tree){
		& & & & \node[term] (u4) {$S_0u^4$}; \\
		& & & \node[nterm] (u3) {$S_0u^3$}; & \\
		& & \node[nterm] (u2) {$S_0u^2$}; & & \node[term] (u3d) {$S_0u^3d$}; \\
		& \node[nterm] (u) {$S_0u$}; & & \node[nterm] (u2d) {$S_0u^2d$};\\
		\node[term] (s) {$S_0$}; & & \node[nterm] (ud) {$S_0ud$}; & & \node[term] (u2d2) {$S_0u^2d^2$}; \\
		& \node[nterm] (d) {$S_0d$}; & &	\node[nterm] (ud2) {$S_0ud^2$};\\
		& & \node[nterm] (d2) {$S_0d^2$}; & & \node[term] (ud3) {$S_0ud^3$}; \\
		& & & \node[nterm] (d3) {$S_0d^3$}; & \\
		& & & & \node[term] (d4) {$S_0d^4$}; \\
	};
	% Lines out of s
	\draw[->] (s) -- (u) node[midway,above] {$p_u$};
	\draw[->] (s) -- (d) node[midway,below] {$p_d$};
	% Lines out of u
	\draw[->] (u) -- (u2) node[midway,above] {$p_u$};
	\draw[->] (u) -- (ud) node[midway,above] {$p_d$};
	% Lines out of d
	\draw[->] (d) -- (ud) node[midway,below] {$p_u$};
	\draw[->] (d) -- (d2) node[midway,below] {$p_d$};
	% Lines out of u2
	\draw[->] (u2) -- (u3) node[midway,above] {$p_u$};
	\draw[->] (u2) -- (u2d) node[midway,above] {$p_d$};
	% Lines out of ud
	\draw[->] (ud) -- (u2d) node[midway,above] {$p_u$};
	\draw[->] (ud) -- (ud2) node[midway,below] {$p_d$};
	% Lines out of d2
	\draw[->] (d2) -- (ud2) node[midway,below] {$p_u$};
	\draw[->] (d2) -- (d3) node[midway,below] {$p_d$};
	% Lines out of u3
	\draw[->] (u3) -- (u4) node[midway,above] {$p_u$};
	\draw[->] (u3) -- (u3d) node[midway,above] {$p_d$};
	% Lines out of u2d
	\draw[->] (u2d) -- (u3d) node[midway,above] {$p_u$};
	\draw[->] (u2d) -- (u2d2) node[midway,above] {$p_d$};
	% Lines out of ud2
	\draw[->] (ud2) -- (u2d2) node[midway,below] {$p_u$};
	\draw[->] (ud2) -- (ud3) node[midway,below] {$p_d$};
	% Lines out of d3
	\draw[->] (d3) -- (ud3) node[midway,below] {$p_u$};
	\draw[->] (d3) -- (d4) node[midway,below] {$p_d$};
	\end{tikzpicture}
	
	\caption{A 4 step lattice ($ N = 4 $)}
	\label{fig:crr-4tr}
\end{figure}





\section{European options}
\label{sec:european-options}

\section{American options}
\label{sec:american-options}


%%% Local Variables:
%%% mode: latex
%%% TeX-master: t
%%% End:
